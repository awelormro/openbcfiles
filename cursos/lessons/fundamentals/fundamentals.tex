\documentclass{article}

\begin{document}
  \begin{center}
    Historia de la programación
  \end{center}
  \begin{enumerate}
    \item Inicio: Década de los 40, con ENIAC, se usaban secuencias de código base, tomando el sistema de palabras, entre 50 y 70 con fortran, lisp y cobol se siguen utilizando, tambiń salió bspl (antecesor a C)
    \item Década de los 70: uso de c y pascal, y el orgigen de sistemas de uso múltiple.
    \item Década de los 80: Inicio de lenguajes de nicho, ADA (defensa), perl(ficheros de texto), bsl(de donde sale tk)
    \item Década de los 90: Lenguajes de propósito múltiple, cosas como java(uso de sistemas embedidos), y javascript(aplicaciones web), python (prototipos,ml, backend) y php.
    \item Décadas posteriores (actionscript con aplicaciones flash) C\# (financial) go (backend) clang y clojure, swift con Apple.
  \end{enumerate}
  \begin{center}
    Tipos de lenguajes de programación
  \end{center}

  \begin{enumerate}
    \item Lenguajes compilados: Aquellos que desde el código se genera un sistema que pueda leer directamente el sistema operativo. Tipos: C, go, ensamblador. Ventajas:
      \begin{itemize}
        \item Más rápido: se manda directo con respecto al procesador.
        \item 
      \end{itemize} 
      Desventajas: 
      \begin{itemize}
        \item No funciona de la misma manera para todas las arquitecturas de procesadores
      \end{itemize}
    \item Lenguajes interpretados: Aquellos que requieren de una herramienta que se llama bytecode, mandando allamar un intérprete que lee el código y lo ejecuta sobre sí mismo, siendo más lento que los compilados. Tipos: java (necesita una java virtual machine), python, perl, php.
      Ventajas: 
      \begin{itemize}
        \item No requiere hacer ajustes para el procesador
      \end{itemize}
      Desventajas:
      \begin{itemize}
        \item más lento
      \end{itemize}
    \item Lenguajes híbridos: Usa un intérprete denominado JIT (just in time)
  \end{enumerate}

  Otra manera de clasificarlos es mediante los tipados o no tipados:
  \begin{enumerate}
    \item Lenguajes tipados: Se indica desde el principio la variable que será
    \item Lenguajes no tipados: El intérprete lee y explica los tipos de variables que se almacenen.
  \end{enumerate}
  Los tipos de aplicaciones pueden ser:
  \begin{enumerate}
    \item escritorio
    \item web: requiere frontend y backend para funcionar, se comunican mediante APIs, el backend puede ir en diversos lenguajes.
    \item móvil: Las apps móviles se desarrollan mediante el uso de vistas, pueden funcionar con frontend y con backend.
  \end{enumerate}
  \begin{center}
    Aplicaciones cliente-servidor
  \end{center}
  Las aplicaciones web se dividen con html-css y se manda la info con API, que comunican al cliente (frontend) con el servidor (backend) aunque se pueden simplificar la escritura de las API con librerías externas
\end{document}
